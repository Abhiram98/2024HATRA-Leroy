\section{Discussion and Future Work}
\label{sec:conc}

Although \toolname generalizes Stitch \cite{Bowers_2023stitch} and addresses 
the main issues Stitch has when fed with python code, it has yet to be more extensively and rigorously tested, 
as well as compared against other previous work. Prior to testing, \toolname can also be extended further to better address 
the stated problems with more advanced program analysis methods, to optimize the current solutions approaches. Furthermore, \toolname extends Stitch without utilising program semantics as babble~\cite{Cao_2023babble} does. Hence, \toolname can be enhanced by adopting from babble and utilizing semantic equivalence to expand the abstraction choices. Lastly, currently \toolname is only tested for compression, but library abstraction for python also targets readability and other program quality metrics that can be encoded in the utility function, and are yet to be tested. 
% \todo{say something about extending from p2 to entire python}



% \begin{enumerate}
%     \item Integration of babble and stitch
%     \item Extension of evaluation
%     \item Comparison against other related work: Regal (novelty), extract-method tools, ast-based code clone detection. Check other work form ICSE/FSE.
%     \item Thoughts about what would make this a full research paper
% \end{enumerate}

% Additionally, \toolname ensures that abstractions involve parameters that are data types.

% Match locations only contain valid function calls.

% Implementation in on ast-completing checking in Rust

% Find a way to measure usefulness/how interesting an abstraction is.